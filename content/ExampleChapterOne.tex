\chapter{Example Chapter One}

\section{First Section}

\subsection{Subsection One}

\begin{figure}
  \centering
  \includegraphics[width=0.8\textwidth]{aurora.jpg}
  \caption{A figure with a caption}
  \label{fig:aurora}
\end{figure}

Figure \ref{fig:aurora} shows a picture.

Citation of a book \cite{hartley2004}.

Let $x$ and $u$ be variables, then:
\begin{equation}
  \frac{d}{dx} \left( \int_{0}^{x} f(u)\,du \right) = f(x) ~.
\end{equation}

\lipsum[1-2]

\subsection{Subsection Two}

\begin{figure}
  \centering
  \input{images/plot.tikz}
  \caption{A tikz plot with a caption}
  \label{fig:plot}
\end{figure}

Figure \ref{fig:plot} shows a plot generated with tikz and pgfplots.

\lipsum[4-5]

\section{Another Section}

\begin{figure}[h]
  \centering
  \subcaptionbox{Picture one}
  {\includegraphics[width=0.4\linewidth]{aurora.jpg}}
  \label{fig:subcaption_1}
  \hfill
  \subcaptionbox{Picture two}
  {\includegraphics[width=0.4\linewidth]{aurora.jpg}}
  \label{fig:subcaption_2}
  \caption{Two pictures side-by-side
    \protect\\ % use this for linebreaks in caption
    New line}
  \label{fig:subcaption}
\end{figure}

\begin{table}
  \centering
  \begin{tabular}{ l | c || r }
    1 & 2 & 3 \\
    4 & 5 & 6 \\
    7 & 8 & 9 \\
  \end{tabular}
  \caption{A table with numbers}
  \label{tab:numbers}
\end{table}

Numbers are shown in table \ref{tab:numbers}

\lipsum[6-12]

% example algorithm taken from algorithm2e's documentation
\begin{algorithm}[H]
  \SetAlgoLined\KwData{this text}
  \KwResult{how to write algorithm with \LaTeX2e }
  initialization\;
  \While{not at end of this document}{read current\;
    \eIf{understand}{go to next section\;current section becomes this one\;}
    {go back to the beginning of current section\;}}
  \caption{How to write algorithms}
\end{algorithm}